\par\noindent\rule{\textwidth}{0.1pt}

В основном пишу на языках низкого уровня типо ассемблера архитектур x86\_64 и базово RISC-V; и C. Поэтому увлекаюсь вопросами встроенных системам и IoT. Среди проектов у меня в основном различные учебные проекты и модули устройств для ядра Linux:
\begin{enumerate}
  \item Символьные, блочные, сетевые модули устройств для ядра Linux (работа проверялась в основном на Ubuntu 20.04-22.04) - \url{https://github.com/Come1LLF00/device-drivers-development}
  \item Агломеративная иерархическая кластеризация на C (\url{https://github.com/Come1LLF00/nuke}), так что да я знаком с некоторыми методами машинного обучения
  \item Многопоточное приложение с использованием Posix Threads (\url{https://gitlab.se.ifmo.ru/system-software/ifmo-spring-2022/posix/come_ill_foo})
  \item Аллокатор памяти с использованием системного вызова mmap (\url{https://gitlab.se.ifmo.ru/come_ill_foo/assignment-memory-allocator})
  \item Сепия фильтр с использованием ассемблерных SIMD инструкций на архитектуре x86\_64 (\url{https://gitlab.se.ifmo.ru/come_ill_foo/assignment-sepia-filter})
  \item Базовые функции ввода-вывода на ассемблере x86\_64 (\url{https://gitlab.se.ifmo.ru/come_ill_foo/assignment-1-io-library})
  \item Разворот картинки в формате BMP на C (\url{https://gitlab.se.ifmo.ru/come_ill_foo/assignment-image-rotation})
  \item Компилятор Pascal-подобного языка, строящий AST дерево и транслирующий его в трёхадресный код. Используются flex и bison (\url{https://github.com/Come1LLF00/student-pc})
  \item Знаком с форматами исполняемых файлов PE и ELF
\end{enumerate}

\par\noindent\rule{\textwidth}{0.1pt}

Знаком с отладочной платой Xilinx Nexys 4 DDR и писал для нее прошивки на Verilog HDL в Vivado Design Suite (также знаком с Icarus Verilog):
\begin{enumerate}
  \item Доработка однотактового RISC-V процессора своими инструкциями (\url{https://github.com/Come1LLF00/schoolRISCV})
  \item Модуль вычисляющий математическую функцию (использует семисегментный индикатор и переключатель для взаимодействия с пользователем) - \url{https://github.com/Come1LLF00/rtl_math_acc}
\end{enumerate}

\par\noindent\rule{\textwidth}{0.1pt}

Также писал CRUD с использованием Java + Spring Boot + Vue.js + PostgreSQL + REST API:
\begin{enumerate}
  \item Фронтенд (\url{https://github.com/Come1LLF00/looking-glass})
  \item Бэкенд (\url{https://github.com/Come1LLF00/alicization})
  \item Скрипты для создания модели в БД (\url{https://github.com/Come1LLF00/alice_in_the_wonderland})
\end{enumerate}

\par\noindent\rule{\textwidth}{0.1pt}

Пользовался различными инструментами тестирования и мониторинга и анализа бинарных файлов:
\begin{enumerate}
  \item JUnit 5 (\url{https://github.com/Come1LLF00/integration-test-functional-system})
  \item Selenium IDE + Selenide (\url{https://github.com/Come1LLF00/fishki-dnet-testing})
  \item Apache JMeter (нагрузочное и стресс-тестирование университетских учебных стендов)
  \item Анализ трафика различных протоколов с Wireshark
  \item Binutils (readelf, objdump, nm)
  \item Различные утилиты Linux (ps, top, pidstat, netstat, tcpdump, ...) - сборник питоновских оберток над стандартными утилитами для удобства сбора метрик во времени и сохранением их в графики \url{https://github.com/Come1LLF00/cifostoolbox}
\end{enumerate}

\par\noindent\rule{\textwidth}{0.1pt}

Помимо просто утилит на Python 3 также по работе пользовался и такими популярными библиотеками, как Celery (брокер и бэкенд Redis в Docker-контейнере), Flask, Socket.IO, SQLAlchemy и JSON-RPC.

\par\noindent\rule{\textwidth}{0.1pt}

С Docker-ом знаком на уровне запуска готовых образов из docker hub и написании простеньких Dockerfile-ов для создания своих образов. Есть небольшой опыт настройки совсем простеньких CI/CD на gitlab (\url{https://gitlab.se.ifmo.ru/come_ill_foo/assignment-sepia-filter}) и на github (\url{https://github.com/Come1LLF00/schoolRISCV/actions}).

\par\noindent\rule{\textwidth}{0.1pt}

Есть проекты по шаблону для STM32F427 и отлаженные на ней же:
\begin{enumerate}
  \item Калькулятор с использованием клавиатуры и oLED-дисплея (\url{https://github.com/Come1LLF00/SDK_Calculator})
  \item Обработка прерываний с таймеров (\url{https://github.com/Come1LLF00/SDK_cLab_interruptions})
\end{enumerate}
Разрабатывались в STM32 Cube IDE.

\par\noindent\rule{\textwidth}{0.1pt}
